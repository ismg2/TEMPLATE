\documentclass[11pt,a4paper]{report}
\usepackage[french]{babel}
\usepackage[utf8]{inputenc}
\usepackage[T1]{fontenc}
\usepackage{layout}
\usepackage{titlesec}
\usepackage{lmodern}
\usepackage{enumitem}
\usepackage{amsmath}
\usepackage{empheq}
\usepackage{amssymb}
\usepackage{mathrsfs}
\usepackage{array}
\usepackage{gensymb}
\usepackage{mathenv}
\usepackage{color}
%\usepackage[intlimits]{amsmath}
\usepackage[usenames,dvipsnames]{xcolor}
\usepackage{listings}
\usepackage{graphicx}
\usepackage{fancybox}
\usepackage{fourier-orns}


\definecolor{codecustom}{rgb}{0,0.5,0}
\definecolor{codegray}{rgb}{0.7,0.7,0.7}
\definecolor{codepurple}{rgb}{0.58,0,0.82}
\definecolor{backcolour}{rgb}{0.975,0.97,0.975}
\definecolor{custom}{RGB}{216,0,127}
\definecolor{lightgray}{gray}{0.75}
\definecolor{myblue}{rgb}{.8, .8, 1}
\definecolor{redcustom}{RGB}{255,0,0}

\titleformat{\chapter}[hang]{\bf\huge}{\thechapter}{2pc}{}


\usepackage[final]{pdfpages} %inclure un pdf


  \newcommand*\mybluebox[1]{%
    \colorbox{myblue}{\hspace{0.5em}#1\hspace{1em}}}

\lstdefinestyle{mystyle}{
	language=VHDL,
    backgroundcolor=\color{backcolour},
    commentstyle=\color{codecustom},
    keywordstyle=\color{custom},
    morekeywords={Fe, Te, sin, carre, Liste, ELEMENT},
    numberstyle=\tiny\color{codegray},
    stringstyle=\color{codepurple},
    basicstyle=\footnotesize,
    breakatwhitespace=false,
    breaklines=true,
    captionpos=t,
    keepspaces=true,
    numbers=left,
    numbersep=5pt,
    showspaces=false,
    showstringspaces=false,
    showtabs=false,
    tabsize=2,
    extendedchars=true,
    literate={é}{{\'e}}1 {Ú}{{\`e}}1 {Ã}{{\`a }}1 {ç}{{\c{c}}}1 {œ}{{\oe}}1 {ÃŜ}{{\`u}}1
{É}{{\'E}}1 {È}{{\`E}}1 {À}{{\`A}}1 {Ç}{{\c{C}}}1 {Œ}{{\OE}}1 {Ê}{{\^E}}1
{ê}{{\^e}}1 {î}{{\^i}}1 {ÃŜ}{{\^o}}1 {û}{{\^u}}1 {â}{{\^a}}1,
}

\lstset{style=mystyle}

\frenchbsetup{StandardItemLabels=true, CompactItemize=false, ReduceListSpacing=true}
\setcounter{tocdepth}{4}
\usepackage{caption}
\DeclareCaptionFont{white}{\color{white}}
\DeclareCaptionFormat{listing}{\colorbox{codecustom}{\parbox{\textwidth}{#1#2#3}}}
\captionsetup[lstlisting]{format=listing,labelfont=white,textfont=white}

\renewcommand{\lstlistingname}{Extrait du code}

%%%%%%%% Espace entre paragraphe et Indentation %%%%%%%%%%%%%%%%%%%%

\setlength{\parskip}{0.09cm}
\setlength{\parindent}{0.8cm}

%%%%%%%%%%%%%%%%%%%%%%%%%%%%%%%%%%%%%%%%%%%%%%%%%%%%%%%%%%%%%%%%%%%%


%%%%%%%%%%%%%%%%%%%%%% Dimensions de la page %%%%%%%%%%%%%%%%%%%%%%%%%

\usepackage[top=2cm, bottom=2cm, left=2cm, right=2cm]{geometry}

%%%%%%%%%%%%%%%%%%%%%%%%%%%%%%%%%%%%%%%%%%%%%%%%%%%%%%%%%%%%%%%%%%%%%%

%%%%%%%%%%%%%%%%%%%% Lien Hyperref Sommainre %%%%%%%%%%%%%%%%%%%%%%%%%%


\usepackage{hyperref} % Créer des liens et des signets

%%%%%%%%%%%%%%%%%%%%%%%%%%%%%%%%%%%%%%%%%%%%%%%%%%%%%%%%%%%%%%%%%%%%%%

%%%%%%%%%%% Gestion des en-tête/pieds de page %%%%%%%%%%%%%%%%%%%%%%%%%%%

\usepackage{fancyhdr}
\pagestyle{fancy}

% Permet d'écrire le nom des chapitres et section en minuscules au lieu de majuscules definit par défaut
\renewcommand{\chaptermark}[1]{\markboth{\bsc{\chaptername~\thechapter{} :} #1}{}}
\renewcommand{\sectionmark}[1]{\markright{\thesection{} #1}}
%

\renewcommand{\headrulewidth}{0.4pt} % epaisseur du trait
\fancyhead[C]{}
\fancyhead[L]{\leftmark}
\fancyhead[R]{Guedira}

\renewcommand{\footrulewidth}{0.4pt}
\fancyfoot[C]{\textbf{Page \thepage}}
%\fancyfoot[L]{}
%\fancyhead[C]{}
\fancyhead[R]{\leftmark}
\fancyhead[L]{}

\renewcommand{\footrulewidth}{0.4pt}
\fancyfoot[C]{\textbf{Page \thepage}}
\fancyfoot[L]{\textsc{Ismail Guedira}}
\fancyfoot[R]{\rightmark}

%%%%%%%%%%%%% Option des annexes %%%%%%%%%%%%%%%%%%%%%%%%%%%

\usepackage[toc,page]{appendix}
\renewcommand{\appendixtocname}{Annexes}
\renewcommand{\appendixpagename}{Annexes}
\renewcommand{\appendixname}{Annexes}
%%%%%%%%%%%%%%%%%%%%%%%%%%%%%%%%%%%%%%%%%%%%%%


\usepackage{pifont} %utiliser des signes pour les itemize plutot cool

\usepackage{lscape}	%pouvoir passer en mode paysage



%%%%%%%%%% Change le nom Table en Tableau %%%%%%%%%%

\addto\captionsfrench{\def\tablename{Tableau}}



\title{Intégration du logiciel Mathcad à la Conception Électronique}
\author{Ismail Guedira}
\usepackage{layout}



\begin{document}


%%%%% %%%%%%%%%                       Et si on refaisait la page de garde            %%%%%%%%%%%%%%%%%%%%%%%%%%%%%%

\makeatletter
  \begin{titlepage}
  \centering
      {%\includegraphics[height=0.09\textheight]{logo_phelma.png}
    \hfill \Large \textbf{Grenoble INP - Phelma}}\\
    \hfill \large \textsc{Physique Électronique Matériaux}\\
    \hfill  3, Parvis Louis Néel \\
	   \hfill  38 000 Grenoble\\

	\hspace{3em}

  %   { \includegraphics[height=0.065\textheight]{logo_Rolls_Royce.jpg} \hfill \Large \textbf{Rolls-Royce}}\\
  %   \hfill \large \textsc{Civil Nuclear SAS - Instrumentation \& Control }\\
	% \hfill 23, Chemin du Vieux Chêne   \\
  %   \hfill 38 246 Meylan  \\


   \vfill

    \hrule
    \vspace{1.5em}
    	{	\LARGE \textsc{RAPPORT DE STAGE} \\
    \vspace{1.5em}
       \LARGE \textbf{\@title}} \\
    \vspace{1.5em}
    \hrule
    \vspace{3 em}
	\Large \textbf{Stagiaire :}		\hfill   \textsc{\@author} \\
                                  	\hfill   \large $2^{ième}$ Année - \textbf{S}ystèmes \textbf{É}lectroniques \textbf{I}ntégrés \\
   									\hfill   \textit{ismail.guedira@phelma.grenoble-inp.fr} \\
	\vspace{2em}

    % \Large \textbf{Maitre de Stage :} 	\hfill	\Large \textsc{Hugo Petit} \\
   % 										\hfill	\large \textbf{Analog Design Manager} \\
   % 										\hfill	\textit{hugo.petit.cn@rolls-royce.com} \\


    \vfill

     {\Large \textsc{2014/2015}  \hfill \Large \textsc{\@date}} \\


  \end{titlepage}
\makeatother

%%%%%%%%%%%%%%%%%%%%%%%%%%%%%%%%%%%%%%%%%%%%%%%%%%%%%%%%%%%%%%%%%%%%%%%%%%%%%%%%%%%%%%%%%%%%%%%%%%%%%%%%%%%%%%%%%%%


\pagenumbering{roman}
\renewcommand{\contentsname}{Sommaire}
\tableofcontents


\newpage
\chapter*{Remerciement}
\addcontentsline{toc}{chapter}{Remerciement}

REMERCIEMENT


\chapter*{Introduction}
\addcontentsline{toc}{chapter}{Introduction}

INTRODUCTION


% \newpage
% 
% \addcontentsline{toc}{chapter}{Abstract}
% \vspace{3cm}
% \textbf{Abstract} \\
% \newline
% \hspace*{17pt}

\listoffigures

\listoftables


\newpage

\pagenumbering{arabic}
\chapter{Mise en contexte}

\chapter{Chapitre 1}

\section{1 - Section 1}

\section{1 - Section 2}

\chapter{Chapitre 2}

\section{2 - Section 1}

\section{2 - Section 2}

\chapter{Chapitre 3}

\section{3 - Section 1}

\section{3 - Section 2}

\section{Je veux mettre du code : Exemple}

\begin{lstlisting}[label=un-label,caption=Titre du code : ex VHDL ]
 sig_A <= "01010101";
 sig_B <= "00110011";		 
 sig_CMD <= "101";			-- case 7: question 6 (test avec erreur pour assert)		   
 wait for delay;
 assert sig_S = "01110101" report "Imagination" severity note;
 assert false report "ERREUR = TRUE" severity note;
\end{lstlisting}


\chapter*{Conclusion}
\addcontentsline{toc}{chapter}{Conclusion}


%%%%%%%%%%%%%%% Début de la gestion de l'annexe %%%%%%%%%%%%%%

%\begin{appendices}
%\include{Annexe/annexe}
%\end{appendices}

%%%%%%%%%%%%%%%%%%%%%%%%%%%%%%%%%%%%%%%%%%%%%%%%%%%%%%%%%%%%



\end{document}
