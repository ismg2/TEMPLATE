% Exemple de CV utilisant la classe moderncv
% Style classic en bleu
% Article complet : http://blog.madrzejewski.com/creer-cv-elegant-latex-moderncv/

\documentclass[11pt,a4paper]{moderncv}
\moderncvtheme[blue]{classic}                
\usepackage[utf8]{inputenc}
\usepackage[top=1.1cm, bottom=1.1cm, left=2cm, right=2cm]{geometry}
% Largeur de la colonne pour les dates
\setlength{\hintscolumnwidth}{2.5cm}

\firstname{Ismail}
\familyname{Guedira}
\title{Élève ingénieur en électronique analogique, hyperfréquence et numérique}              
\address{Route de Cossonay 112}{1008 Prilly - Suisse}    
\email{ismail.guedira@phelma.grenoble-inp.fr}                      
%\homepage{www.madrzejewski.com}
\mobile{+33 6 22 45 51 93} 
\photo[64pt][0.4pt]{photo}
%\quote{Stage blabla}
%\extrainfo{21 ans -- Permis B}
\begin{document}
\maketitle

\section{Formations}
\cventry{2013 -- Actuel}{Grenoble INP - Phelma (Institut National Polytechnique)}{}{}{}{
\begin{itemize}
\item Semestre d'échange à l'EPFL (École Polytechnique Fédérale de Lausanne).
\item Filière Systèmes Électroniques Intégrés, spécialisée dans la microélectronique numérique et analogique, l'électronique hyperfréquence et les systèmes optoélectroniques.%\newline{}
\end{itemize}
%\newline{}
}
\cventry{2011 -- 2013}{Classes Préparatoires aux Grandes Écoles}{Lycée Descartes}{Tours}{}{Première année en MPSI (Maths Physique et Sciences de l'Ingénieur) et seconde année en PSI* (Physique Sciences de l'Ingénieur).}

\section{Expérience}
\cventry{Sept. 2015\\à Aujourd'hui}{Conception d'une réference bandgap, EPFL}{Neuchâtel}{Suisse}{}{
Projet de semestre au sein de l'ESPLAB (Laboratoire d'Électronique et de Traitement du Signal), il consiste à concevoir une réference de tension extrêmement peu sensible aux variations de température et de la tension d'alimentation.}

\cventry{Mai 2015\\à Août 2015}{Rolls-Royce Civil Nuclear Instrumentation \& Control}{Meylan}{France}{\newline{}Stage d'application (4 mois)}{Intégration du logiciel Mathcad dans le processus de développement électronique.\\ 
%Pour faciliter la réutilisation des données de conception des blocs électroniques au sein d'une nouvelle carte, il était nécessaire de mettre en place une méthode adapté à l'équipe de conception électronique de Rolss-Royce pour travailler efficacement sur Mathcad.
En collaboration avec l'ensemble du pôle de conception électronique, nous avons établi une méthode et une syntaxe de travail efficace sur Mathcad.}

\cventry{Février 2015\\à Avril 2015}{Conception d'un convertisseur, Grenoble INP - Phelma}{Grenoble}{France}{}
{Lors de ce projet en micro-électronique mixte analogique et numérique, j'ai pu parcourir les différentes phases de conception d'un convertisseur analogique-numérique de l'étude système et de la caractérisation complète du circuit jusqu'à la réalisation du layout. %un cahier des charges pré-établi.
}
\cventry{Août 2014}{ST Microelectronics}{Crolles}{France}{\newline{}Stage Opérateur (5 semaines)}{Stage opérateur en salle blanche à l'usine Crolles 200 mm en équipe de nuit, préparation des wafers tests pour les différentes unités de production de l'usine.
%\newline{}
}
\cventry{Février 2014\\à Février 2015}{Vice-President à la Junior Conseil Phelma}{}{}{}{%
\begin{itemize}
  \item Gestion d'une équipe de 38 personnes,
  \item Organisation du Forum des Entrerpises de Grenoble INP - Phelma,
  \item Prospection des entreprises et négociations de partenariats,
  \item Recrutement et formation de la nouvelle équipe à la prospection et à la communication,
  \item Représentation de l'association lors de différentes conférences et forums.
\end{itemize}
}


\section{Compétences}
\cvitem{Informatique}{Programmation en langage C : Réalisation d'un émulateur de microprocesseur MIPS}
\cvitem{Logiciels}{Modélisation et simulation de circuits analogique et numérique avec \textbf{Cadence}, de circuits de communications numériques avec \textbf{Matlab} et \textbf{Simulink} et de circuits micro-ondes avec \textbf{ADS}}
%\cvitem{MATLAB\\et Simulink}{Simulation de circuit de communications numériques}
%\cvitem{ADS}{Simulation de circuit micro-ondes}
%\cvitem{Langues}{Anglais -- titulaire du BULATS (niveau C1, 83/100), notions d'Espagnol}
% \cvitem{Administration}{Apache2, BIND, Postfix, Fail2ban, Proxmox (openVZ), Iptables,Nagios}
% \cvitem{Réseaux}{Réseaux WAN, Protocole IP, VLAN, Téléphonie IP}
% \cvitem{Logiciels}{Microsoft office, Open Office, Adobe Photoshop}
\cvlanguage{Anglais}{lu, écrit, parlé -- BULATS niveau C1 83 / 100}{}

%\cvcomputer{Langages}{(X)HTML, PHP, CSS, PL/SQL, C/C++, Java, Bash}{}{}
%\cvcomputer{Base de données}{MySQL, Oracle}{CMS}{Wordpress, Symfony2 (notions)}
%\cvcomputer{Analyse}{Merise, UML, Design Patterns}{}{}
%\cvcomputer{Systèmes}{Windows XP/Seven/Server 2003/Server 2008, Linux (Debian)}{}{}
%\cvcomputer{Administration}{Apache2, BIND, Postfix, Fail2ban, Proxmox (openVZ), Iptables,Nagios}{}{}

\section{Centres d'intérêt}
\cvitem{Basket-Ball}{Pratique régulière en club pendant 10 ans et dans le cadre du sport universitaire actuellement}
%\cvitem{Entreprenariat}{Passive Income, The 4 hour work week}
%\cvitem{Plongée}{Diplôme de niveau 1}

\end{document}

